\section{Substitutions}

\frame{\tableofcontents[currentsection]}

\begin{frame}
  \frametitle{Substitutions}
  \begin{itemize}
    \item Say you have a large string
    \item You want to replace all occurrences of a certain pattern
    \item Examples
      \begin{itemize}
        \item Remove all comments from source code
        \item Add anchor tags around email addresses in HTML
        \item Change the format of dates
      \end{itemize}
  \end{itemize}
\end{frame}

\begin{frame}
  \frametitle{Example Problem}
  \begin{center}
    Write a function \texttt{remove\_comments(string)} that removes comments from source code.
  \end{center}
  \vskip5mm
  \structure{Rules}\\[2mm]
  A comment can be recognized by
  \begin{itemize}
    \item Everything after \texttt{//} until the end of a line is comment
    \item Everything between \texttt{/* */} is comment
  \end{itemize}
\end{frame}

\begin{frame}
  \frametitle{Solution}
  \structure{Regex Under Construction}
  \begin{center} \ttfamily
    \only<1>{//}
    \only<2>{//\alert{.*}}
    \only<3>{//.*\alert{\$}}
    \only<4>{//.*\alert{?}\$}
    \only<5>{//.*?\$\alert{|}}
    \only<6>{//.*?\$|\alert{/*}}
    \only<7>{//.*?\$|/\alert{\textbackslash}*}
    \only<8>{//.*?\$|/\textbackslash*\alert{.*}}
    \only<9>{//.*?\$|/\textbackslash*.*\alert{\textbackslash*/}}
    \only<10>{//.*?\$|/\textbackslash*.*\alert{?}\textbackslash*/}
  \end{center}
  \vskip5mm
  \structure{Explanation}
  \vskip2mm
  \begin{overprint}
    \onslide<1>
    \begin{center}
      We start with the literal \texttt{//}
    \end{center}

    \onslide<2>
    \begin{center}
      Anything that follows is comment \\
      This matches everything from \texttt{//} up to the end of the string
    \end{center}

    \onslide<3>
    \begin{center}
      \texttt{\$} is an anchor: anchors indicate a position between characters \\
      In this case, \texttt{\$} matches the location at the end of the string \\
      We can also make \texttt{\$} match the end of a line (see later), which is what we'll rely on here
    \end{center}

    \onslide<4>
    \begin{center}
      \texttt{.*} is said to be \emph{greedy}: it matches as much as possible \\
      This causes the regex to match from \texttt{//} to the \emph{last} newline \\
      By adding a \texttt{?} we make it \emph{non-greedy} \\
      This causes the matching to stop as soon as a newline is found
    \end{center}

    \onslide<5>
    \begin{center}
      We introduce an alternative patting
    \end{center}

    \onslide<6>
    \begin{center}
      It starts with \texttt{/*}
    \end{center}

    \onslide<7>
    \begin{center}
      Uh oh! \\
      \texttt{/*} means "zero or more slashes". \\
      We want it to take the \texttt{*} literally. \\
      We need to \emph{escape} the \texttt{*} by adding a \texttt{\textbackslash}.
    \end{center}

    \onslide<8>
    \begin{center}
      Any character can be comment
    \end{center}

    \onslide<9>
    \begin{center}
      We end the pattern with \texttt{*/}
    \end{center}

    \onslide<10>
    \begin{center}
      We run into trouble again with the greediness of \texttt{.*}. \\
      For example, in \\ \texttt{/* x */ aaa /* y */} \\ it matches from the first \texttt{/*} to the \emph{last} \texttt{*/}, \\ thereby including \texttt{aaa} in the comment. \\
      We need to make it non-greedy.
    \end{center}
  \end{overprint}
\end{frame}

\begin{frame}
  \frametitle{Python in Action}
  \code[language=python3,width=.95\linewidth,font=\small]{sub.py}
  \begin{itemize}
    \item In order to allow \texttt{/* */} comments spread over multiple lines, the \texttt{.} must also match newlines
    \item By default, \texttt{.} does \emph{not} match newlines
    \item Adding \texttt{re.DOTALL} makes \texttt{.} match newlines
    \item Adding \texttt{re.MULTILINE} makes \texttt{\$} match just before a newline
  \end{itemize}
\end{frame}

\begin{frame}
  \frametitle{Example Problem}
  \begin{center}
    Write a function \texttt{fix\_dates} that changes the format of dates.
  \end{center}
  \vskip5mm
  \structure{Rules}\\[2mm]
  \begin{itemize}
    \item A date is written as month/day/year
    \item Month, day and year consist of digits
    \item A date must be rewritten as day/month/year
  \end{itemize}
\end{frame}

\begin{frame}
  \frametitle{Python in Action}
  \code[language=python3]{fix-dates.py}
  \begin{itemize}
    \item The substitution depends on what is matched
    \item The replacement string should be "smart"
    \item We can use \texttt{\textbackslash1}, \texttt{\textbackslash2}, \dots to refer to capturing groups
  \end{itemize}
\end{frame}