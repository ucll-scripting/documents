\section{Find All}

\frame{\tableofcontents[currentsection]}

\begin{frame}
  \frametitle{Find All}
  \begin{itemize}
    \item Say you have a large string
    \item You want to find all parts that satisfy a certain pattern
    \item Examples
      \begin{itemize}
        \item Find all email addresses in a text document
        \item Find all links in an html document
        \item Find all web server request coming from a certain IP
      \end{itemize}
  \end{itemize}
\end{frame}

\begin{frame}
  \frametitle{Example Problem}
  \begin{center}
    Write a function \texttt{find\_email\_addresses} that looks for all email addresses in a document.
  \end{center}
  \vskip5mm
  \structure{Rules}
  \begin{itemize}
    \item An email address consists of two parts: a user name and a domain name.
    \item The user name and domain name are separated by an \texttt{@}.
    \item The user name consists of one or more dot-separated nonempty alphanumeric strings.
    \item The domain name consists of two or more dot-separated nonempty alphanumeric strings.
  \end{itemize}
\end{frame}

\begin{frame}
  \frametitle{Solution}
  \structure{Regex Under Construction}
  \begin{center}\ttfamily\small
    \only<1>{[a-zA-Z0-9]}
    \only<2>{[a-zA-Z0-9]\alert{+}}
    \only<3>{[a-zA-Z0-9]+\alert{\textbackslash.[a-zA-Z0-9]+}}
    \only<4>{[a-zA-Z0-9]+\alert{(}\textbackslash.[a-zA-Z0-9]+\alert{)*}}
    \only<5>{[a-zA-Z0-9]+(\textbackslash.[a-zA-Z0-9]+)*\alert{@}}
    \only<6>{[a-zA-Z0-9]+(\textbackslash.[a-zA-Z0-9]+)*@\alert{[a-zA-Z0-9]+(\textbackslash.[a-zA-Z0-9]+)+}}
  \end{center}
  \vskip5mm
  \structure{Explanation}
  \vskip2mm
  \begin{overprint}
    \onslide<1>
    \begin{center}
      A single alphanumeric character can be
      \begin{itemize}
        \item a lowercase letter
        \item an uppercase letter
        \item a digit
      \end{itemize}
    \end{center}

    \onslide<2>
    \begin{center}
      An alphanumeric string contains one or more alphanumeric characters
    \end{center}

    \onslide<3>
    \begin{center}
      We add a second alphanumeric string which is separated from the first by a dot
    \end{center}

    \onslide<4>
    \begin{center}
      This extra part (dot + string) can be repeated zero or more times
    \end{center}

    \onslide<5>
    \begin{center}
      We add the \texttt{@}
    \end{center}

    \onslide<6>
    \begin{center}
      We repeat the user name pattern, except we require the \texttt{.alphanum+} part
      is repeated at least once.
    \end{center}
  \end{overprint}
\end{frame}

\begin{frame}
  \frametitle{Python in Action}
  \code[language=python3,width=.95\linewidth,font=\small]{findall.py}
\end{frame}

\begin{frame}
  \frametitle{Bug!}
  \code[language=python3,width=.95\linewidth,font=\small]{findall-bug.py}
  \vskip5mm
  \begin{overprint}
    \onslide<1>
    \begin{center}
      Why these weird results?
    \end{center}
    \onslide<2>
    \begin{center}
      Capturing groups make \texttt{findall} think we're \\ only interested
      in those parts \\[2mm]
      We would prefer the complete email addresses \\ instead of just parts
    \end{center}
  \end{overprint}
\end{frame}

\begin{frame}
  \frametitle{Python in Action}
  \code[language=python3,width=.95\linewidth,font=\small]{findall-fixed.py}
  \begin{center}
    By adding \texttt{?:} just after \texttt{(}, \\
    we tell Python it's not a capturing group \\[2mm]
    {\tiny we warned you it was cryptic}
  \end{center}
\end{frame}

\begin{frame}
  \frametitle{Checking}
  \code[language=python3,width=.95\linewidth,font=\small]{findall-fixed-use.py}
\end{frame}
