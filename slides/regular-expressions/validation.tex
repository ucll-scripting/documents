\section{Validation}

\frame{\tableofcontents[currentsection]}

\begin{frame}
  \frametitle{Validation}
  \begin{itemize}
    \item Checking if a string has a valid structure
    \item User input validation
    \item Examples
      \begin{itemize}
        \item Is this string a valid integer number?
        \item Is this string a valid floating point number?
        \item Is this string a valid date?
        \item Is this string a valid time?
        \item Is this string a valid email address?
      \end{itemize}
  \end{itemize}
\end{frame}

\begin{frame}
  \frametitle{Example Problem \#1}
  \begin{center}
    Write a function \texttt{is\_valid\_float(string)} that checks if \texttt{string} is a valid floating point number.
  \end{center}
  \vskip5mm
  \structure{Rules}
  \begin{itemize}
    \item \texttt{string} must only contain the following characters
      \begin{itemize}
        \item One or more digits
        \item At most one minus sign, which if present must be in front
        \item At most one dot
      \end{itemize}
    \item If a dot occurs, there must be at least one digit preceding it and one following it
  \end{itemize}
\end{frame}

\begin{frame}
  \frametitle{Possible Implementation}
  \code[language=python3,font=\small,width=.95\linewidth]{is-valid-float.py}
\end{frame}

\begin{frame}
  \frametitle{Example Problem \#2}
  \begin{center}
    Write a function \texttt{is\_valid\_date(string)} that checks if \texttt{string} is a valid date.
  \end{center}
  \vskip5mm
  \structure{Rules}
  \begin{itemize}
    \item A date consist of three numbers separated by a separator
    \item The first number must have exactly 2 digits
    \item The second number must have exactly 2 digits
    \item The third number must have exactly 2 or 4 digits
    \item The separator must be either \texttt{-} or \texttt{/}
    \item The same separator must be used twice
    \item We don't care about the actual values of days or months (e.g., we allow month 54)
  \end{itemize}
\end{frame}

\begin{frame}
  \frametitle{Possible Implementation}
  \begin{center}
    Left as an exercise to the reader
  \end{center}
\end{frame}

\begin{frame}
  \frametitle{Easier Solution?}
  \begin{itemize}
    \item Checking for validity can become complicated
    \item Alternative: regular expressions
  \end{itemize}
\end{frame}

\begin{frame}
  \frametitle{Regular Expressions}
  \begin{itemize}
    \item A regular expression represents a pattern
    \item We can ask "does this string satisfy this pattern?"
    \item Regular expressions are a Domain Specific Language
      \begin{itemize}
        \item Ultra specialized in one task
        \item Regular expressions can become quite cryptic
        \item Made for succinctness, not readability
      \end{itemize}
  \end{itemize}
\end{frame}

\begin{frame}
  \frametitle{Literal Pattern}
  \structure{Regular Expression}
  \begin{center}
    \texttt{abcd}
  \end{center}
  \structure{Meaning}
  \begin{center}
    String must be equal to \texttt{abcd} exactly
  \end{center}
  \structure{Example}
  \begin{overprint}
    \onslide<1>
    \begin{center}
      String \texttt{"abcd"} matches the pattern \\[2mm]
      All letters appear in the right order
    \end{center}

    \onslide<2>
    \begin{center}
      String \texttt{"abc"} does not match the pattern \\[2mm]
      \texttt{d} is missing
    \end{center}

    \onslide<3>
    \begin{center}
      String \texttt{"dabc"} does not match the pattern \\[2mm]
      Letters appear in wrong order
    \end{center}

    \onslide<4-5>
    \begin{center}
      String \texttt{"ABCD"} does not match the pattern \\[2mm]
      Regexes are case sensitive by default
    \end{center}
  \end{overprint}

  \visible<5>{
    \begin{center}
      \tikz{
        \node[fill=blue!50,draw] {
          \parbox{8cm}{\centering
            Regex is actually overkill in this case \\
            Simple comparison is enough
          }
        }
      }
    \end{center}
  }
\end{frame}

\begin{frame}
  \frametitle{Repetition}
  \structure{Regular Expression}
  \begin{center}
    \texttt{a*}
  \end{center}
  \structure{Meaning}
  \begin{center}
    String must containing zero or more \texttt{a}s
  \end{center}
  \structure{Example}
  \begin{overprint}
    \onslide<1>
    \begin{center}
      String \texttt{"aaaaa"} matches the pattern \\[2mm]
    \end{center}

    \onslide<2>
    \begin{center}
      String \texttt{"a"} matches the pattern \\[2mm]
    \end{center}

    \onslide<3>
    \begin{center}
      String \texttt{""} matches the pattern \\[2mm]
      Zero \texttt{a}s is allowed
    \end{center}

    \onslide<4>
    \begin{center}
      String \texttt{"aaaax"} does not match the pattern \\[2mm]
      Only \texttt{a}s are allowed
    \end{center}
  \end{overprint}
\end{frame}

\begin{frame}
  \frametitle{Repetition}
  \structure{Regular Expression}
  \begin{center}
    \texttt{ab*c}
  \end{center}
  \structure{Meaning}
  \begin{center}
    String must start with \texttt{a}, then followed by zero or more \texttt{b}s and end with a \texttt{c}
  \end{center}
  \structure{Example}
  \begin{overprint}
    \onslide<1>
    \begin{center}
      String \texttt{"ac"} matches the pattern \\[2mm]
    \end{center}

    \onslide<2>
    \begin{center}
      String \texttt{"abc"} matches the pattern \\[2mm]
    \end{center}

    \onslide<3>
    \begin{center}
      String \texttt{"abbbbbbc"} matches the pattern \\[2mm]
    \end{center}

    \onslide<4>
    \begin{center}
      String \texttt{"bbbbc"} does not match the pattern \\[2mm]
      Missing \texttt{a}
    \end{center}

    \onslide<5>
    \begin{center}
      String \texttt{"a"} does not match the pattern \\[2mm]
      Missing \texttt{c}
    \end{center}

    \onslide<6>
    \begin{center}
      String \texttt{"aabababc"} does not match the pattern \\[2mm]
      Only \texttt{b} can be repeated
    \end{center}
  \end{overprint}
  \visible<6>{
    \begin{center}
      \tikz{
        \node[fill=blue!50,draw] {
          \parbox{8cm}{\centering
            Note how the \texttt{*} only applies to the character directly preceding it
          }
        }
      }
    \end{center}
  }
\end{frame}

\begin{frame}
  \frametitle{Parentheses}
  \structure{Regular Expression}
  \begin{center}
    \texttt{(ab)*c}
  \end{center}
  \structure{Meaning}
  \begin{center}
    String must start with zero or more \texttt{ab}s, followed by \texttt{c}
  \end{center}
  \structure{Example}
  \begin{overprint}
    \onslide<1>
    \begin{center}
      String \texttt{"abc"} matches the pattern \\[2mm]
    \end{center}

    \onslide<2>
    \begin{center}
      String \texttt{"ababc"} matches the pattern \\[2mm]
    \end{center}

    \onslide<3>
    \begin{center}
      String \texttt{"c"} matches the pattern \\[2mm]
    \end{center}

    \onslide<4>
    \begin{center}
      String \texttt{"ac"} does not match the pattern \\[2mm]
      \texttt{a} on its own is not enough
    \end{center}
  \end{overprint}
\end{frame}

\begin{frame}
  \frametitle{Alternatives}
  \structure{Regular Expression}
  \begin{center}
    \texttt{a|b}
  \end{center}
  \structure{Meaning}
  \begin{center}
    String must be equal to either \texttt{a} or \texttt{b}
  \end{center}
  \structure{Example}
  \begin{overprint}
    \onslide<1>
    \begin{center}
      String \texttt{"a"} matches the pattern \\[2mm]
    \end{center}

    \onslide<2>
    \begin{center}
      String \texttt{"b"} matches the pattern \\[2mm]
    \end{center}

    \onslide<3>
    \begin{center}
      String \texttt{"ab"} does not match pattern \\[2mm]
      Either \texttt{a} or \texttt{b}, but not both
    \end{center}
  \end{overprint}
\end{frame}

\begin{frame}
  \frametitle{Character Class}
  \structure{Regular Expression}
  \begin{center}
    \only<1-4>{
      \texttt{(0|1|2|3|4|5|6|7|8|9)*}
    }
    \only<5>{
      \texttt{[0123456789]*}
    }
    \only<6>{
      \texttt{[0-9]*}
    }
    \only<7>{
      \texttt{\textbackslash d}
    }
  \end{center}
  \structure{Meaning}
  \begin{center}
    String must be equal to zero or more digits
  \end{center}
  \structure{Example}
  \begin{overprint}
    \onslide<1>
    \begin{center}
      String \texttt{"1"} matches the pattern \\[2mm]
    \end{center}

    \onslide<2>
    \begin{center}
      String \texttt{"555"} matches the pattern \\[2mm]
    \end{center}

    \onslide<3>
    \begin{center}
      String \texttt{"79856312"} matches the pattern \\[2mm]
    \end{center}

    \onslide<4-7>
    \begin{center}
      String \texttt{"78a31"} does not match pattern \\[2mm]
      \texttt{a} is not allowed
    \end{center}
  \end{overprint}
\end{frame}

\begin{frame}
  \frametitle{Optional, At Least One}
  \structure{Regular Expression}
  \begin{center}
    \texttt{-?[0-9]+}
  \end{center}
  \structure{Meaning}
  \begin{center}
    The string begins with an optional minus sign, followed
    by one or more digits
  \end{center}
  \structure{Example}
  \begin{overprint}
    \onslide<1>
    \begin{center}
      String \texttt{"1"} matches the pattern \\[2mm]
    \end{center}

    \onslide<2>
    \begin{center}
      String \texttt{"-1"} matches the pattern \\[2mm]
    \end{center}

    \onslide<3>
    \begin{center}
      String \texttt{"764516"} matches the pattern \\[2mm]
    \end{center}

    \onslide<4>
    \begin{center}
      String \texttt{"77-"} does not match pattern \\[2mm]
      Minus sign in the wrong place
    \end{center}

    \onslide<5>
    \begin{center}
      String \texttt{"-"} does not match pattern \\[2mm]
      Needs at least one digit
    \end{center}
  \end{overprint}
\end{frame}

\begin{frame}
  \frametitle{Any Character, Number of Occurrences}
  \structure{Regular Expression}
  \begin{center}
    \texttt{.{8,16}}
  \end{center}
  \structure{Meaning}
  \begin{center}
    The string consists of 8 to 16 arbitrary characters
  \end{center}
  \structure{Example}
  \begin{overprint}
    \onslide<1>
    \begin{center}
      String \texttt{"12345678"} matches the pattern \\[2mm]
    \end{center}

    \onslide<2>
    \begin{center}
      String \texttt{"1234567890ABCDEF"} matches the pattern \\[2mm]
    \end{center}

    \onslide<3>
    \begin{center}
      String \texttt{"1234567"} does not match the pattern \\[2mm]
      Too few characters
    \end{center}

    \onslide<4>
    \begin{center}
      String \texttt{"1234567890ABCDEFG"} does not match pattern \\[2mm]
      Too many characters
    \end{center}
  \end{overprint}
\end{frame}

\begin{frame}
  \frametitle{Example Problem \#1: Reminder}
  \begin{center}
    Write a function \texttt{is\_valid\_float(string)} that checks if \texttt{string} is a valid floating point number.
  \end{center}
  \vskip5mm
  \structure{Rules}
  \begin{itemize}
    \item \texttt{string} must only contain the following characters
      \begin{itemize}
        \item One or more digits
        \item At most one minus sign, which if present must be in front
        \item At most one dot
      \end{itemize}
    \item If a dot occurs, there must be at least one digit preceding it and one following it
  \end{itemize}
\end{frame}

\begin{frame}
  \frametitle{Example Problem \#1: Using A Regular Expression}
  \structure{Regex Under Construction}
  \begin{center}\ttfamily
    \only<1>{-}
    \only<2>{-\alert{?}}
    \only<3>{-?\alert{\textbackslash d}}
    \only<4>{-?\textbackslash d\alert{+}}
    \only<5>{-?\textbackslash d+\alert{.}}
    \only<6>{-?\textbackslash d+\alert{\textbackslash}.}
    \only<7>{-?\textbackslash d+\textbackslash.\alert{\textbackslash d+}}
    \only<8>{-?\textbackslash d+\alert{(}\textbackslash.\textbackslash d+\alert{)?}}
  \end{center}
  \vskip5mm
  \structure{Explanation}
  \vskip2mm
  \begin{overprint}
    \onslide<1>
    \begin{center}
      We start with the minus sign
    \end{center}
    \onslide<2>
    \begin{center}
      The minus sign is optional, so we add the \texttt{?}
    \end{center}
    \onslide<3>
    \begin{center}
      Next comes a digit
    \end{center}
    \onslide<4>
    \begin{center}
      We allow multiple digits, but there needs to be at least one
    \end{center}
    \onslide<5>
    \begin{center}
      Next comes a dot
    \end{center}
    \onslide<6>
    \begin{center}
      Problem: \texttt{.} has a special meaning (= any character) \\
      In order to express a literal dot, we need to "escape it"
    \end{center}
    \onslide<7>
    \begin{center}
      After the dot, one or more digits must appear
    \end{center}
    \onslide<8>
    \begin{center}
      The dot and the digits following it are optional
    \end{center}
  \end{overprint}
\end{frame}

\begin{frame}
  \frametitle{Example Problem \#2: Reminder}
  \begin{center}
    Write a function \texttt{is\_valid\_date(string)} that checks if \texttt{string} is a valid date.
  \end{center}
  \vskip5mm
  \structure{Rules}
  \begin{itemize}
    \item A date consist of three numbers separated by a separator
    \item The first number must have exactly 2 digits
    \item The second number must have exactly 2 digits
    \item The third number must have exactly 2 or 4 digits
    \item The separator must be either \texttt{-} or \texttt{/}
    \item The same separator must be used twice
    \item We don't care about the actual values of days or months (e.g., we allow month 54)
  \end{itemize}
\end{frame}

\begin{frame}
  \frametitle{Example Problem \#2: Using A Regular Expression}
  \structure{Regex Under Construction}
  \begin{center}\ttfamily
    \only<1>{\textbackslash d\textbackslash d}
    \only<2>{\textbackslash d\textbackslash d\alert{-}}
    \only<3>{\textbackslash d\textbackslash d-\alert{\textbackslash d\textbackslash d}}
    \only<4>{\textbackslash d\textbackslash d-\textbackslash d\textbackslash d\alert{-}}
    \only<5>{\textbackslash d\textbackslash d-\textbackslash d\textbackslash d-\textbackslash d\textbackslash d(\textbackslash d\textbackslash d)?}
    \only<6>{\alert{(}\textbackslash d\textbackslash d-\textbackslash d\textbackslash d-\textbackslash d\textbackslash d(\textbackslash d\textbackslash d)?\alert{)|(\textbackslash d\textbackslash d/\textbackslash d\textbackslash d/\textbackslash d\textbackslash d(\textbackslash d\textbackslash d)?)}}
    \only<7>{\textbackslash d\textbackslash d\alert{([-/])}\textbackslash d\textbackslash d\alert{\textbackslash1}\textbackslash d\textbackslash d(\textbackslash d\textbackslash d)?}
  \end{center}
  \vskip5mm
  \structure{Explanation}
  \vskip2mm
  \begin{overprint}
    \onslide<1>
    \begin{center}
      Two digits for the day
    \end{center}
    \onslide<2>
    \begin{center}
      Separator, let's for now assume \texttt{-}
    \end{center}
    \onslide<3>
    \begin{center}
      Two digits for the month
    \end{center}
    \onslide<4>
    \begin{center}
      Second separator
    \end{center}
    \onslide<5>
    \begin{center}
      Two or four digits for year
    \end{center}
    \onslide<6>
    \begin{center}
      Same pattern but with separator \texttt{/}
    \end{center}
    \onslide<7>
    \begin{center}
      The \texttt{\textbackslash1} is a backreference \\
      It says "here needs to be the same as in the first group" \\
      A group is formed using \texttt{()}, in this case it is \texttt{([-/])}
    \end{center}
  \end{overprint}
\end{frame}

\begin{frame}
  \frametitle{Usage in Python}
  \code[language=python3,width=.95\linewidth]{fullmatch.py}
\end{frame}
