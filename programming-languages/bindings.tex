\section{Language Bindings}

\frame{\tableofcontents[currentsection]}

\begin{frame}
  \frametitle{Operating System}
  \structure{Services Provided by OS}
  \begin{itemize}
    \item File System
    \item Graphical User Interface
    \item Networking
    \item Multimedia
    \item \dots
  \end{itemize}
\end{frame}

\begin{frame}
  \frametitle{APIs}
  \begin{itemize}
    \item \link{https://docs.microsoft.com/en-us/windows/win32/apiindex/windows-api-list}{Windows API}, \link{https://www.kernel.org/doc/html/v4.12/core-api/index.html}{Linux API}
    \item Windows and Linux are written in C
    \item OS services provided as C functions
    \item Easily accessible if we use C for our own programs
    \item But what if we use another language?
  \end{itemize}
  \begin{overprint}
    \onslide<1>
    \begin{center}
      \begin{tikzpicture}[layer/.style={fill=blue!50,minimum width=4cm,minimum height=1cm,draw,anchor=south west,drop shadow},
                          note/.style={draw,fill=green!50,opacity=.75,font=\small},
                          arrow/.style={-latex,thick}]
        \path[use as bounding box] (0,0) rectangle (4,4);
        \node[layer] (os) at (0,0) { OS };
        \node[layer] (program) at (0,2) { Program };
        \node[note,anchor=south west] (os note) at ($ (os.north east) + (0.25,0.25) $) { Written in C };
        \node[note,anchor=south west] (program note) at ($ (program.north east) + (0.25,0.25) $) { Written in C };
        \draw[arrow] (os note.west) to[bend right=30] (os);
        \draw[arrow] (program note.west) to[bend right=30] (program);
        \draw[-latex,thick] (program.south) -- (os.north) node[midway,left,font={\small\ttfamily}] { playSound(sound); };
      \end{tikzpicture}
    \end{center}

    \onslide<2>
    \begin{center}
      \begin{tikzpicture}[layer/.style={fill=blue!50,minimum width=4cm,minimum height=1cm,draw,anchor=south west,drop shadow},
                          note/.style={draw,fill=green!50,opacity=.75,font=\small},
                          arrow/.style={-latex,thick}]
        \path[use as bounding box] (0,0) rectangle (4,4);
        \node[layer] (os) at (0,0) { OS };
        \node[layer] (program) at (0,2) { Program };
        \node[note,anchor=south west] (os note) at ($ (os.north east) + (0.25,0.25) $) { Written in C };
        \node[note,anchor=south west] (program note) at ($ (program.north east) + (0.25,0.25) $) { Not written in C };
        \draw[arrow] (os note.west) to[bend right=30] (os);
        \draw[arrow] (program note.west) to[bend right=30] (program);
        \draw[-latex,thick] (program.south) -- (os.north) node[midway,left,font={\small\ttfamily}] { ??? };
      \end{tikzpicture}
    \end{center}
  \end{overprint}
\end{frame}

\begin{frame}
  \frametitle{Language Bindings}
  \begin{itemize}
    \item Most languages allow \emph{foreign functions}
    \item Behave like regular functions
    \item Are implemented in another language (probably C)
    \item All necessary "translations" are done behind the scenes
    \item Java examples
      \begin{itemize}
        \item \link{https://github.com/openjdk/jdk/blob/master/src/java.base/share/classes/java/lang/System.java\#L415}{\texttt{System.currentTimeMillis()}}
        \item \link{https://github.com/openjdk/jdk/blob/master/src/java.base/share/classes/java/io/FileInputStream.java}{Reading files}
        \item \link{https://github.com/openjdk/jdk/blob/master/src/java.base/share/classes/java/lang/Runtime.java}{Memory Usage}
      \end{itemize}
  \end{itemize}
\end{frame}

\begin{frame}
  \frametitle{Python}
  \begin{itemize}
    \item Many functions/classes/\dots are actually written in C
    \item Python code glues these together
    \item Speed of Python execution does not matter
    \item Most time is spent in C code
    \item Similar to military hierarchy
      \begin{itemize}
        \item General has overview, can take his time
        \item Soldiers perform the work, must be efficient
      \end{itemize}
  \end{itemize}
  \begin{center}
    \begin{tikzpicture}[base/.style={circle,minimum size=2mm,inner sep=0mm},
                        python/.style={base,fill=blue!50},c/.style={base,fill=red!50},
                        arrow/.style={-latex}]
      \node[python] (root) {};
      \node[python] (a1) at ($ (root) + (-1.25,-0.5) $) {};
      \node[python] (a2) at ($ (root) + (1.25,-0.5) $) {};
      \draw[arrow] (root) -- (a1);
      \draw[arrow] (root) -- (a2);
      \foreach[evaluate={int(\i*2-1)} as \j,evaluate={int(\j+1)} as \k] \i in {1,2} {
        \node[c] (b\j) at ($ (a\i) + (-0.5,-0.5) $) {};
        \node[c] (b\k) at ($ (a\i) + (0.5,-0.5) $) {};
        \draw[arrow] (a\i) -- (b\j);
        \draw[arrow] (a\i) -- (b\k);
      }
      \foreach[evaluate={int(\i*2-1)} as \j,evaluate={int(\j+1)} as \k] \i in {1,...,4} {
        \node[c] (c\j) at ($ (b\i) + (-0.25,-0.5) $) {};
        \node[c] (c\k) at ($ (b\i) + (0.25,-0.5) $) {};
        \draw[arrow] (b\i) -- (c\j);
        \draw[arrow] (b\i) -- (c\k);
      }
      \foreach[evaluate={int(\i*2-1)} as \j,evaluate={int(\j+1)} as \k] \i in {1,...,8} {
        \node[c] (d\j) at ($ (c\i) + (-0.125,-0.5) $) {};
        \node[c] (d\k) at ($ (c\i) + (0.125,-0.5) $) {};
        \draw[arrow] (c\i) -- (d\j);
        \draw[arrow] (c\i) -- (d\k);
      }

      \draw[|-|] let \p1=(root) in (3,\y1) -- ++(0,-0.5) node[midway,right,font=\tiny] {Python, 10\% of the time};
      \draw[|-|] let \p1=(a1) in (3,\y1) -- ++(0,-1.5)   node[midway,right,font=\tiny] {C, 90\% of the time};
    \end{tikzpicture}
  \end{center}
\end{frame}

\begin{frame}
  \frametitle{Example}
  \code[language=python3]{sum.py}
  \begin{itemize}
    \item Our own \texttt{my\_sum} definition written in Python
    \item Built-in \texttt{sum} function written in C
    \item Both do exactly the same thing
    \item \texttt{sum} is 4-5$\times$ faster
  \end{itemize}
\end{frame}